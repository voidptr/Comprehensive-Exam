\subsection{What is Robustness? }

Much like evolvability, robustness is a set of overlapping concepts concerned with the ability of a genotype to maintain a given phenotype despite an unexpected disruption [ref].  Most commonly, robustness is studied in regards to either perturbations in the environment or else mutational disruptions. In the first case, the evolution of robustness to environmental disturbances depends heavily on the flexibility and decoupling of gene regulatory or signaling pathways[ref]. For example, a gene-regulatory or signaling pathway that is loosely coupled may make use of signaling from multiple incoming paths, rather than depending on a single, rigid precursor. This type of arrangement is more likely to continue to function even if some part of the signaling path is disrupted. An example of this kind of robust arrangement is nerve conduction in vertebrates where axons connect several cells, thus routing signals in parallel, and avoiding single points of failure. [ref].
For the purposes of my research, I will focus on the second case: mutational robustness. Distinct from robustness to environmental perturbation, robustness against mutation depends largely on degeneracy, redundancy, and regulatory decoupling [ref]. Degeneracy refers to a many-to-one relationship between an encoding and a product, such that several codes can produce a single output [ref]. Thus, there is a chance that mutations in the code will not alter the product. One example of this feature is codon degeneracy in biological organisms, where, depending on the hydropathy of the amino-acid, single, or even double mutations in some positions of the encoding do not affect the binding of the encoded amino-acid [ref].
Similarly, redundancy refers to the duplication of function in multiple places in the genome, such that mutations altering function in one copy of a gene do not alter function in the other copy [ref]. Redundancy may also refer to redundancy of function within genes, such that if a mutation occurs in one portion of a gene, other neighboring portions of the protein will compensate, and the protein will retain its structure and function. [ref]
Finally, regulatory decoupling allows for more than one kind regulatory precursor to provide inputs for a process [ref]. Thus, if mutation were to damage one set of precursors, others can take their place and preserve function. An example of this kind of architecture is in the production of the acetate precursor for the Krebs cycle, which produces ATP in all aerobic organisms [ref]. Acetate can be derived from either carbohydrates, lipids, or proteins, thus if any of those pathways are damaged by mutation, acetate can still be produced from other sources, and ATP production can continue.
It is worth noting that many of the architectural features that confer robustness to processes and genomes are based on arrangements of modular structures [ref]. In this way, much of robustness is facilitated by the evolution of modularity.

