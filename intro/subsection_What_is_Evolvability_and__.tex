\subsection{What is Evolvability (and why is it hard to pin down)}

In its most abstract sense, evolvability appears to boil down to a simple concept: the ability of genetic systems to produce adaptive variation. However, the devil is in the details.  How, exactly do genetic systems generate adaptive variation? How do we measure this potential? Should all forms of variation count as evolvability? At what time-scales does evolvability act?  And finally, how did it evolve? Are evolvable features under direct selection, or are they by-products of other processes?

Evolvability, in its details, must mean different things at different evolutionary scopes and timescales. Depending on your perspective, evolvability can describe the response to selection at the population level \cite{Fisher_1930}\cite{houle_comparing_1992}, to larger phenomena such as variability generation [wagner/albrecht], exploration of neutral spaces and robustness[ref? Kitano?], generation of novel features [ref ?], or even the potential to generate the larger clade-level innovations[Kirschner/Gerhart] and major transitions[ref Maynard-Smith David Ackley]. Beyond that, there is lots of confusion and controversy about the definitions and components of evolvability even within a certain scope. [ref pigliucci paper]

Finally, it is unclear whether evolvability is acted upon by direct selection, or whether it is a byproduct of other traits that are selected upon, or some combination of the two. At the individual level, it’s possible that some traits that support evolvability, such as robustness of developmental [KirchnerGerhart96] or cell processes[ref?] could have been selected for directly in response to adverse environmental conditions. However, at the population level, traits like neutral variation generation could have piggybacked on the genomes of the adaptive variants that they produced. Finally, at the clade level, genetic structures that produced populations of adaptive variants with robust and flexible genetic architectures would have been more successful at adaptive radiations [ Dawkins], and thus go on to found whole branches of life with those traits [KG]. 

However, we must use caution when invoking selection at higher levels than the individual. While there is some evidence to support clade-level selection in the evolution of evolvability [ref? look in pigluicci] caution should be applied when attributing evolutionary outcomes to higher levels of selection when random chance or  lower levels are adequately explanatory. Caution should be used to avoid falling into the trap of adaptationism [ref spendrels paper] by assuming that evolvability is an end in itself. Selection can only act on organisms and populations as they exist, and it is an error to automatically assume that patterns identified in hind-sight are predictive of future evolution.

With so many scopes and levels of evolvability described in the literature, with varying levels of detail and predictive power, it may be best to avoid attempting to unify the concept, and rather acknowledge that “evolvability” is not a singular idea, but rather an overlapping and interrelated set of concepts relating to adaptation and evolutionary potential. In order to understand the large field of evolvability and how the distinct scopes and ideas connect together, a historical narrative may be useful.

