\section{Introduction - Change, Adaptation and the Evolution of Evolvability}
\subsection{Evolvability and Evolutionary Potential - Why Study It} 
Evolutionary potential and its evolution are entwined in many of the biggest questions of evolutionary biology. The evolution of sex, multi-cellularity, and other major transitions are characterized by significant changes in genetic architecture that appear to have facilitated the transitions. \cite{szathmary_major_1995} The adaptive radiations that accompanied metazoan evolution \cite{kirschner_evolvability_1998}, were also accompanied by changes in genetic architecture that were carried along as species diversified and colonized new ecological niches. The vast diversity of species and their complex ecological interplay depends fundamentally on the ability of populations to not only adapt to their environment, but also to create, explore, and exploit ecological niches. Evolvability and genetic architecture are so fundamental, that much of evolutionary biology ceases to make sense without them. 

Within evolutionary computation, evolvability is also fundamental. The representation problem, which influences every aspect of evolutionary search, can be characterized as a problem of how to structure the genetic code such that genomes can not only express complex solutions, but can also be mutated in meaningful ways. \cite{dawkins_13_2003} By definition, systems that exhibit good characteristics in evolvability produce better solutions more quickly, while avoiding premature convergence.\verb| [ref]| Beyond the representation problem, many of the barriers to complexity are actually barriers to evolvability.

