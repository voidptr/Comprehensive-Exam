\subsection{So, what do I mean by Evolvability?}

As I described above, evolvability is a series of distinct, but overlapping concepts that are generally concerned with adaptation, variation, and/or novelty generation. For the purposes of my research, I am using the Wagner/Albrecht conception of evolvability, which focuses on variability, that is, the generation of adaptive variation in response to mutation. Variability depends primarily on the organization and interrelation of the components of the genome; that is, the genetic architecture, and the resulting genotype-to-phenotype map.

The major features that influence this metric for evolvability appear to be modularity of functional components and phenotypic robustness to mutation and environmental perturbation. While there are other architectural features that also contribute to evolvability, they will not be the focus of this proposal.

