\subsection{Avida}

We use the Avida digital evolution platform to examine the various ways by which evolvability, modularity, and robustness evolves, and how these mechanisms affect the course of evolution.
Avida is a software platform for performing evolution experiments with digital organisms in a virtual world. The organisms are composed of a circular genome of assembly-like computer instructions that are executed in a virtual CPU. Populations of these organisms are placed in a toroidal world in individual cells, where they are allowed to execute, reproduce, compete for space, mutate and evolve. 
 
Populations in Avida experience natural selection. The Avida world is not of infinite size, but space constrained. This means that organisms are continuously competing for space in which to reproduce.  Avida allows the experimenter to subject the organisms to various kinds of selective pressures, including providing them with activities that may be performed in exchange for a boost to execution speed that gives the organisms a competitive advantage. But even without explicit external pressures, organisms still experience an implicit pressure to execute more quickly and/or efficiently. The organisms that run fastest are able to reproduce more quickly, and thus outcompete their peers for space.  
Populations are also subject to mutation. The genome in the initial default organism contains all the instructions necessary for reproduction. However, the instructions are not executed with perfect fidelity. By default, the reproductive copy instruction is faulty, meaning that it will probabilistically introduce errors (mutations) into the offspring genomes. These offspring organisms carry and execute the mutations to their genomes, and in turn pass them on, along with new mutations to their own offspring.
In order to perform the experiments described in later chapters, both code and configuration modifications were made to Avida, including modifications to mutation operators, sexual reproduction, resource uptake and utilization, and reward mechanisms. Each of these modifications are described in more detail in later chapters.
Avida is available for download from XXXXX, and specific versions along with datafiles to reproduce the experiments described in this paper may be found at ZZZZZZ.