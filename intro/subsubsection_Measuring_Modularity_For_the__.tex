\subsubsection{Measuring Modularity}

For the purposes of this research, I will focus on spatial modularity. Spatial modularity may be measured by calculating the proportion of traits that are affected by a given site in the genome, normalized by the number of sites that code for a trait \cite{misevic_sexual_2006}. Trivially, this can be measured by performing knock-out experiments to identify the sites that contribute to particular function.

\begin{quote}
Spatial Modularity, $m_S$, is measured by:

\begin{enumerate}
\item counting the total number of traits expressed in a genome: $T$
\item enumerate and count the number of sites that code for any trait: set $K, k$
\item counting the number of traits coded for by each site within set $K: t_k$;
\item calculating the inverse of the average number of traits coded for per site to reflect the level of spatial modularity of coding regions of a genome
\end{enumerate}
\begin{equation}
m_S =  \frac{1}{\frac{1}{k} {\sum_{i=1}^{k} \frac{t_{k}}{T}}} 
\end{equation}
\end{quote}

