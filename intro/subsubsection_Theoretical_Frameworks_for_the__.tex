\subsubsection{Theoretical Frameworks for the Evolution of Evolvability}

The Wagner and Altenberg paper on the evolution of evolvability significantly expanded the theoretical framework behind the evolution of the genotype/phenotype map. [ref WagnerAltenberg96] They draw on knowledge from computational evolution to inform their perspective on evolvability, since the problem of evolvability is central to the representation problem in evolutionary computer science. 

Initially, Wagner and Altenberg emphasized the distinction between variation and variability. Variation is the realized diversity in a population, which is a concept that lies firmly within population genetics and the gene-centric modern synthesis.  Variability is a concept that they introduced to describe the ability to generate new phenotypes in response to mutation or environmental change. Variability is a metric associated with a local neighborhood in a genotype to phenotype map, and depends on features of that map, including pleiotropy and modularity, and robustness and flexibility of biological processes.

Wagner and Altenberg focused in particular on modularity as a key feature that contributes to the variability or evolvability of a generative system. They defined modularity as the functional separation of distinct trait-complexes. Modularity supports flexible and robust developmental and cell processes that are resistant to environmental perturbation and mutation. [ref the paper that W/A reference here] Modularity also facilitates easier adaptation because it minimizes pleiotropic bonds between unrelated traits that would otherwise slow adaptation. Thus, indirect selection for modularity is feasible at the individual level due to the advantage it provides by making the developmental process more robust to perturbation.

Wagner and Altenberg’s paper led to a vast proliferation of new work exploring the evolution of evolvability. Of particular note is the Kirschner and Gerhart 1996 paper, which explored metazoan evolution for examples of traits that, in combination, acted to increase evolvability. The authors found numerous examples of new, evolvable features coinciding with adaptive radiations. The authors also develop a case for a combination of direct selection upon the individual for evolvability-enhancing features, and those traits persisting as by-products as a result of adaptive radiations, setting the stage for the evolution of more and more complex evolvable features.

