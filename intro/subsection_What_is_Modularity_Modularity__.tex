\subsection{What is Modularity?}

Modularity is the degree to which traits or genes are self-contained. From the perspective of the phenotype, modularity can be defined as the degree to which traits are decoupled from each other, and are able to evolve and optimize freely\verb|[ref]|. Phenotypic (functional) modularity may therefore be considered to be the inverse of pleiotropy \verb|[ref]|.

From the perspective of the genome, modularity is defined as the degree to which functional gene regions are distinct from each other \verb|[ref]|. An increase in spatial modularity is the diminishment of overlap of gene regions that code for a particular function. Spatial modularity is highly correlated with functional modularity [ref], though it is possible to have spatially modular genomes that are not functionally modular, and vice-versa.\verb|[ref]|

Features such as evolvability and robustness rely heavily on spatial and functional modularity \verb|[ref]|. Traits with high functional modularity, and thus low pleiotropy, are able to evolve independently, and respond more quickly to selection, thus contributing to evolvability \verb|[ref]|. Additionally, functionally modular traits, because they are more independent, may be more easily repurposed or co-opted by other traits to add new function. \verb|[ref]|  Conversely, spatially modular genomic regions, because they are more self-contained, tend to better resist disruption from recombination \cite{misevic_sexual_2006} and horizontal gene transfer \verb|[ref HGT papers?]|, thus increasing robustness.

