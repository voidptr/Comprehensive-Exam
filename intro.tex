\section{Introduction - Change, Adaptation and the Evolution of Evolvability}
\subsection{Evolvability and Evolutionary Potential - Why Study It} 
Evolutionary potential and its evolution are entwined in many of the biggest questions of evolutionary biology. The evolution of sex, multi-cellularity, and other major transitions are characterized by significant changes in genetic architecture that appear to have facilitated the transitions. [Maynard Smith and Szathmary 1995] The adaptive radiations that accompanied metazoan evolution [Kirschner/Gerhard 1996], were also accompanied by changes in genetic architecture that were carried along as species diversified and colonized new ecological niches. The vast diversity of species and their complex ecological interplay depends fundamentally on the ability of populations to not only adapt to their environment, but also to create, explore, and exploit ecological niches. Evolvability and genetic architecture are so fundamental, that much of evolutionary biology ceases to make sense without them. 

Within evolutionary computation, evolvability is also fundamental. The representation problem, which influences every aspect of evolutionary search, can be characterized as a problem of how to structure the genetic code such that genomes can not only express complex solutions, but can also be mutated in meaningful ways. [ref Dawkins] By definition, systems that exhibit good characteristics in evolvability produce better solutions more quickly, while avoiding premature convergence. [ref] Beyond the representation problem, many of the barriers to complexity are actually barriers to evolvability.

\subsection{What is Evolvability (and why is it hard to pin down)}

In its most abstract sense, evolvability appears to boil down to a simple concept: the ability of genetic systems to produce adaptive variation. However, the devil is in the details.  How, exactly do genetic systems generate adaptive variation? How do we measure this potential? Should all forms of variation count as evolvability? At what time-scales does evolvability act?  And finally, how did it evolve? Are evolvable features under direct selection, or are they by-products of other processes?

Evolvability, in its details, must mean different things at different evolutionary scopes and timescales. Depending on your perspective, evolvability can describe the response to selection at the population level [fisher, houle], to larger phenomena such as variability generation [wagner/albrecht], exploration of neutral spaces and robustness[ref? Kitano?], generation of novel features [ref ?], or even the potential to generate the larger clade-level innovations[Kirschner/Gerhart] and major transitions[ref Maynard-Smith David Ackley]. Beyond that, there is lots of confusion and controversy about the definitions and components of evolvability even within a certain scope. [ref pigliucci paper]

Finally, it is unclear whether evolvability is acted upon by direct selection, or whether it is a byproduct of other traits that are selected upon, or some combination of the two. At the individual level, it’s possible that some traits that support evolvability, such as robustness of developmental [KirchnerGerhart96] or cell processes[ref?] could have been selected for directly in response to adverse environmental conditions. However, at the population level, traits like neutral variation generation could have piggybacked on the genomes of the adaptive variants that they produced. Finally, at the clade level, genetic structures that produced populations of adaptive variants with robust and flexible genetic architectures would have been more successful at adaptive radiations [ Dawkins], and thus go on to found whole branches of life with those traits [KG]. 

However, we must use caution when invoking selection at higher levels than the individual. While there is some evidence to support clade-level selection in the evolution of evolvability [ref? look in pigluicci] caution should be applied when attributing evolutionary outcomes to higher levels of selection when random chance or  lower levels are adequately explanatory. Caution should be used to avoid falling into the trap of adaptationism [ref spendrels paper] by assuming that evolvability is an end in itself. Selection can only act on organisms and populations as they exist, and it is an error to automatically assume that patterns identified in hind-sight are predictive of future evolution.

With so many scopes and levels of evolvability described in the literature, with varying levels of detail and predictive power, it may be best to avoid attempting to unify the concept, and rather acknowledge that “evolvability” is not a singular idea, but rather an overlapping and interrelated set of concepts relating to adaptation and evolutionary potential. In order to understand the large field of evolvability and how the distinct scopes and ideas connect together, a historical narrative may be useful.

\subsection{Historical Conceptions}
\subsubsection{Modern Synthesis}

The evolution of evolvability as a formalized theory originated with Dawkins[ref] and Albrech[ref], though the concept of evolvability (as the response to selection, measured by heritability) existed much earlier, in the work of Fisher [ref] and Wright [ref??].  Fisher’s fundamental theorem of the response of a population to selection identified narrow-sense heritability ($h^2$) as a measure for how evolvable populations were. Evolvability as heritability ($h^2$) is a measure of the portion of the phenotypic variation in a population that can be accounted for by additive genetic effects.

$h_2 = \frac{Var_A}{Var_P}$

While broad-sense heritability ($H^2$) refers to the entire genetic contribution to a population’s variance, including dominance and epistasis, the additive genetic variance is the component that directly relates to a population’s response to selection.[ref … Houle?] As a measure of evolvability, narrow-sense heritability ($h^2$) was also used as a term in the breeder’s equation, in order to estimate the response of a population to artificial selection.

$R = {h^2}S$

Heritability, however, is not an ideal predictor for the response to selection because it fails to integrate factors such as the population distribution of variability in a trait. [ref, from Hansen and Houle’s paper] Heritability, being scaled by total population variation () in a trait would predict the same response to selection regardless of whether the standard deviation of variance of that trait was large or small, or where the mean of that trait lay.
Houle proposed an alternative measure that suffers from fewer of these problems: the Coefficient of Genetic Variation.

\begin{quote}
show formula
\end{quote}

The Coefficient of Genetic Variation ($CV_A$) is superior to narrow-sense heritability because it scales additive genetic variance by the trait mean, rather than the total population variation. Thus, the additive variation component isn’t swamped by large population trait variance. [ref Hansen 2008]. Since life-history (fitness-related) traits tend to have very large population variances [ref], $h^2$predicts that life-history traits have very low heritability and thus low response to selection. $CV_A$, however, being scaled by trait mean, predicts much higher response to selection for life history traits. [ref  Hansen’s “Heritability is not evolvability” 2008] [ref Houle’s paper on CVa, 1992]

Heritability and Houle’s $CV_A$ still suffers from significant drawbacks as predictors for adaptation and evolvability in a larger sense [ref Houle heritability is not evolvability 2008] Both $h^2$ and $CV_A$ measures predict the response to selection based on the expressed trait variation in a population, under the current environmental conditions. They say nothing of the potential for cryptic variation that may be revealed by mutation, nor do they address differences in genetic architecture that may promote faster adaptation. Ultimately, $CV_A$ is best when examining the short-term response to selection in artificially selected populations [ref], in static environments, without a high mutational load.

Clearly, such short-term, population-based measures are unsuitable for measuring larger patterns of the evolution of evolvability, especially over the long term.

\subsubsection{Evolvability as a Distinct Concept}

Dawkins, in his foundational paper on evolvability and evolutionary constraint [ref], reframed the problem of evolvability in the context of computational evolution and development.  Dawkins described a generative genetic system based on a few alleles, and rules that governed development based on the traits encoded in the alleles.  Each allele would govern the execution of a generative rule, and the rules would interact with each other as they produced the phenotype. As he added new kinds of rules (constraints) into the generative process, the system would produce more and more complexity.

Dawkins used this example to draw parallels to biological generative developmental systems and how evolutionary constraints in development allow for more complex and robust phenotypes. Dawkins identified a few key themes that underlay the more powerful features of developmental systems. These systems would be organized in such a way as to facilitate cumulative effects. That is, innovations in constraints can build upon each other and are “cumulative in evolutionarily interesting ways”[ref]

Dawkins theorized that these kinds of generative developmental systems, or “embryologies” were the basis for evolvability, and that they must have evolved as a result of their intrinsic power to produce adaptive variation. Dawkins suggested that the genetic systems that persisted were those that facilitated adaptive radiations into new or otherwise empty ecological niches.

Alberch followed up Dawkins’ ideas with a much more thorough accounting of how, exactly, these kinds of evolvable traits translate into an analyzable phenotype space. Alberch dismantled the concept of a simplistic, hierarchical genotype to phenotype mapping function and emphasized that developmental and cell metabolic systems are strongly dynamical, nonlinear systems, for which genes are just one part of the regulatory cycle. Because of the dynamic nature of cell processes, it was clear that the gene-centric, population genetics view was inadequate to describe the complexity of the processes involved, and how they translated complex parameters into phenotypes. To that end, a new framework for analysis was required.

Alberch introduced the concept of “parameter spaces” to describe the variation in genotypic parameters that result in distinct phenotypes, while addressing the lack of one-to-one correlation between alleles (parameters) and phenotype. Parameter spaces are multidimensional spaces, divided by parameter thresholds (bifurcation boundaries) that form boundaries between phenotypes.  The domains bounded by these thresholds include all of the parameter combinations that produce a given phenotype. Larger domains can be described as more stable than smaller domains, because there are larger ranges of neutral variation available before organisms tip into a different phenotype. Populations with distinct phenotypes and varying parameters can thus be visualized as blobs occupying areas in parameter space.

Alberch contended that the “evolvability potential” of a dynamical system is encapsulated by the properties of the parameter space. Specifically, the topology of the bifurcation boundaries govern the ease with which the systems can produce both neutral and adaptive variation. Alberch asserted that the generative systems must have undergone selection that favors those systems that provide a good balance between exploration and stability, but provided no mechanism for that selection.

Dawkins and Alberch laid out a compelling case for the role generative developmental systems in facilitating evolvability, but their theoretical frameworks were far from complete.

\subsubsection{Theoretical Frameworks for the Evolution of Evolvability}

The Wagner and Altenberg paper on the evolution of evolvability significantly expanded the theoretical framework behind the evolution of the genotype/phenotype map. [ref WagnerAltenberg96] They draw on knowledge from computational evolution to inform their perspective on evolvability, since the problem of evolvability is central to the representation problem in evolutionary computer science. 

Initially, Wagner and Altenberg emphasized the distinction between variation and variability. Variation is the realized diversity in a population, which is a concept that lies firmly within population genetics and the gene-centric modern synthesis.  Variability is a concept that they introduced to describe the ability to generate new phenotypes in response to mutation or environmental change. Variability is a metric associated with a local neighborhood in a genotype to phenotype map, and depends on features of that map, including pleiotropy and modularity, and robustness and flexibility of biological processes.

Wagner and Altenberg focused in particular on modularity as a key feature that contributes to the variability or evolvability of a generative system. They defined modularity as the functional separation of distinct trait-complexes. Modularity supports flexible and robust developmental and cell processes that are resistant to environmental perturbation and mutation. [ref the paper that W/A reference here] Modularity also facilitates easier adaptation because it minimizes pleiotropic bonds between unrelated traits that would otherwise slow adaptation. Thus, indirect selection for modularity is feasible at the individual level due to the advantage it provides by making the developmental process more robust to perturbation.

Wagner and Altenberg’s paper led to a vast proliferation of new work exploring the evolution of evolvability. Of particular note is the Kirschner and Gerhart 1996 paper, which explored metazoan evolution for examples of traits that, in combination, acted to increase evolvability. The authors found numerous examples of new, evolvable features coinciding with adaptive radiations. The authors also develop a case for a combination of direct selection upon the individual for evolvability-enhancing features, and those traits persisting as by-products as a result of adaptive radiations, setting the stage for the evolution of more and more complex evolvable features.

\subsection{So, what do I mean by Evolvability?}

As I described above, evolvability is a series of distinct, but overlapping concepts that are generally concerned with adaptation, variation, and/or novelty generation. For the purposes of my research, I am using the Wagner/Albrecht conception of evolvability, which focuses on variability, that is, the generation of adaptive variation in response to mutation. Variability depends primarily on the organization and interrelation of the components of the genome; that is, the genetic architecture, and the resulting genotype-to-phenotype map.

The major features that influence this metric for evolvability appear to be modularity of functional components and phenotypic robustness to mutation and environmental perturbation. While there are other architectural features that also contribute to evolvability, they will not be the focus of this proposal.

\subsection{What is Modularity?}

Modularity is the degree to which traits or genes are self-contained. From the perspective of the phenotype, modularity can be defined as the degree to which traits are decoupled from each other, and are able to evolve and optimize freely[ref]. Phenotypic (functional) modularity may therefore be considered to be the inverse of pleiotropy [ref].

From the perspective of the genome, modularity is defined as the degree to which functional gene regions are distinct from each other [ref]. An increase in spatial modularity is the diminishment of overlap of gene regions that code for a particular function. Spatial modularity is highly correlated with functional modularity [ref], though it is possible to have spatially modular genomes that are not functionally modular, and vice-versa.[ref]

Features such as evolvability and robustness rely heavily on spatial and functional modularity [ref]. Traits with high functional modularity, and thus low pleiotropy, are able to evolve independently, and respond more quickly to selection, thus contributing to evolvability [ref]. Additionally, functionally modular traits, because they are more independent, may be more easily repurposed or co-opted by other traits to add new function. [ref]  Conversely, spatially modular genomic regions, because they are more self-contained, tend to better resist disruption from recombination [ref misevic] and horizontal gene transfer [ref HGT papers?], thus increasing robustness.

\subsubsection{Measuring Modularity}

For the purposes of this research, I will focus on spatial modularity. Spatial modularity may be measured by calculating the proportion of traits that are affected by a given site in the genome, normalized by the number of sites that code for a trait [ref misevic]. Trivially, this can be measured by performing knock-out experiments to identify the sites that contribute to particular function.

Spatial Modularity, $m_s$, is measured by:

\begin{enumerate}
\item counting the total number of traits expressed in a genome: $T$
\item enumerate and count the number of sites that code for any trait: set $K, k$
\item counting the number of traits coded for by each site within set $K: t_k$;
\item calculating the inverse of the average number of traits coded for per site to reflect the level of spatial modularity of coding regions of a genome [ref ??? Sex in Avida Modularity work?]
\end{enumerate}

$m_{S} =  \frac{1}{\frac{1}{k} {\sum_{i=1}^{k} \frac{t_{k}}{T}}}$ 


\subsection{What is Robustness? }

Much like evolvability, robustness is a set of overlapping concepts concerned with the ability of a genotype to maintain a given phenotype despite an unexpected disruption [ref].  Most commonly, robustness is studied in regards to either perturbations in the environment or else mutational disruptions. In the first case, the evolution of robustness to environmental disturbances depends heavily on the flexibility and decoupling of gene regulatory or signaling pathways[ref]. For example, a gene-regulatory or signaling pathway that is loosely coupled may make use of signaling from multiple incoming paths, rather than depending on a single, rigid precursor. This type of arrangement is more likely to continue to function even if some part of the signaling path is disrupted. An example of this kind of robust arrangement is nerve conduction in vertebrates where axons connect several cells, thus routing signals in parallel, and avoiding single points of failure. [ref].
For the purposes of my research, I will focus on the second case: mutational robustness. Distinct from robustness to environmental perturbation, robustness against mutation depends largely on degeneracy, redundancy, and regulatory decoupling [ref]. Degeneracy refers to a many-to-one relationship between an encoding and a product, such that several codes can produce a single output [ref]. Thus, there is a chance that mutations in the code will not alter the product. One example of this feature is codon degeneracy in biological organisms, where, depending on the hydropathy of the amino-acid, single, or even double mutations in some positions of the encoding do not affect the binding of the encoded amino-acid [ref].
Similarly, redundancy refers to the duplication of function in multiple places in the genome, such that mutations altering function in one copy of a gene do not alter function in the other copy [ref]. Redundancy may also refer to redundancy of function within genes, such that if a mutation occurs in one portion of a gene, other neighboring portions of the protein will compensate, and the protein will retain its structure and function. [ref]
Finally, regulatory decoupling allows for more than one kind regulatory precursor to provide inputs for a process [ref]. Thus, if mutation were to damage one set of precursors, others can take their place and preserve function. An example of this kind of architecture is in the production of the acetate precursor for the Krebs cycle, which produces ATP in all aerobic organisms [ref]. Acetate can be derived from either carbohydrates, lipids, or proteins, thus if any of those pathways are damaged by mutation, acetate can still be produced from other sources, and ATP production can continue.
It is worth noting that many of the architectural features that confer robustness to processes and genomes are based on arrangements of modular structures [ref]. In this way, much of robustness is facilitated by the evolution of modularity.

\subsubsection{Measuring robustness}

Robustness to mutation can be assessed in different ways, either from the perspective of the phenotype, the genotype, or combinations of the two.  From the perspective of an individual genotype, you can assess its individual robustness by calculating the proportion of the number of unique phenotypes that are different from the expressed phenotype, and are in its 1-neighborhood, that is, are connected to that genotype by single-step mutations. [ref]
Genotypic robustness, , of a genotype G, is measured by:
enumerating all possible single-step mutants that may arise from the given genotype: ;
counting those mutants that are neutral phenotypic variants:;
calculating the proportion of neutral phenotypic variants to reflect the probability of a neutral variant being produced by this genotype in response to mutation. [ref A. Wagner a paradox resolved]
r_{G} =  \frac{R_{G}}{nm}
 

Genotypic robustness is trivially negatively correlated with genotypic evolvability, because each neutral variant in the 1-neighborhood of a genotype is, by definition, not of a different phenotype. However, the inverse is not necessarily the case, because each non-neutral neighbor phenotype may not be unique. Therefore, a non-robust genotype may not necessarily have high evolvability if its neighborhood is dominated by a single or few distinct phenotypes [ref].
From the perspective of the phenotype, robustness may be assessed by to taking the average genotypic robustness across the phenotype.
Phenotypic robustness, , is measured by:
counting the number of distinct neutral genetic variants that produce a given phenotypic trait in a population, (k);
calculating the proportion of neutral variants produced by single-step mutations , averaged over all of the neutral genetic variants to reflect the probability of a neutral genotype in the population producing another neutral genotype in response to mutation. [ref A. Wagner a paradox resolved]

r_{P} =  \frac{1}{k} \sum_{i=1}^{k} r_{G} 


Unlike genotypic robustness, higher phenotypic robustness has been shown to correlate with phenotypic evolvability[ref AWagner] in cases where the possible number of neutral variants in a phenotype (the frequency of the phenotype) is high. With increasing numbers of neutral variants, the number of potential unique phenotypes in the 1-neighborhood of the phenotype increases.
These measures of robustness are each limited in that they do not address realized population composition, the shape of the mutational landscape, nor the expected frequency of the target phenotype. In particular, the correlation of phenotypic robustness with evolvability depends on the expected phenotypic frequency [ref]. Thus, if the frequency is unknown, phenotypic robustness may not predict evolvability.
Further, different populations may have vastly different numbers of realized neutral variants for a given phenotype. [reference Alberch, phenotype spaces] Factors such as gene-flow, bottle-necking, linkage dis-equilibrium, founder effects, and sexual selection may strongly affect overall diversity in populations [ref], including the neutral diversity for a particular phenotypic trait that we are concerned with.
For this reason, while population level metrics may cause a phenotype to appear to be non-robust, this apparent value may be the result of the amount and type of realized diversity present in a given population, rather than the robustness of that phenotype as predicted by its potential neutral network [ref alberch]. Thus, because evolvability depends at least in part on realized neutral variation and on expected trait frequency,  is not a good predictor for evolvability.

\subsection{Measuring/Defining Evolvability}

As alluded to above, the features that confer robustness may also promote evolvability by allowing for greater neutral genetic diversity within a given phenotype. The larger the number of distinct genotypes with the same phenotype, the more exploration of the genotype space that can be done without decreasing organismal fitness. As a population diffuses through such a neutral region, more potential phenotypes become available in few mutational steps. [ref neutrality paper]
Historically, predicting this robust-yet-evolvable quality has been challenging. Previously-used measures for robustness that focus on counting the proportion of unique genotypes that compose a phenotype (Phenotypic Robustness [ref Andreas Wagner]) are limited in their ability to predict the evolvability of a population, especially where phenotypic frequency is unknown. 
In contrast, Genomic Diffusion Rate, or the probability that an offspring will express a neutral or positive fitness effect, may be used to characterize overall population evolvability [ref to evolvability in genomic diffusion rate]. However, while genomic diffusion rate is predictive of evolvability, it does not distinguish between types of robustness, and limited in its ability to mechanistically explain the evolvability characteristics of a phenotype in a given population. 
In order to address both of these issues, we will measure evolvability using a series of network graph metrics that describe both the level and shape of interconnectedness of realized neutral variants and their potential mutants. These aggregate metrics will be theoretically and experimentally related to existing robustness and evolvability measures to predict the evolvability of a population by accounting for both the probability of mutations changing phenotypes, as well as addressing the reasons behind those probabilities.

\subsubsection{Characterizing Evolvability with Mutational Network Graphs}

The components of a phenotype can be conceptualized as a graph of connected genotype nodes in the mutational landscape. The evolutionary properties of a phenotype are dependent on the characteristics of this graph.  Each node within the phenotype graph is a member genotype, and it is connected to the other nodes in the phenotype by single-step mutations. For convenience, these nodes are colored white, and identified as realized genotypes. Further, potential mutants in the 1-neighborhood that also share the phenotype are also added as nodes to this graph, but are colored black. Finally, the rest of the 1-neighborhood is added to the graph, colored uniquely based on their phenotype. 
In order to characterize the shape of the mutational landscapes within a phenotype, and their corresponding relationship to evolvability, we will measure the properties of the realized genotype mutational graph, such as average and median node-connectedness, both within and outside the phenotype, both to realized and potential mutants. 
Average Network Density, , is measured by:
enumerate all white nodes (set K)
count the proportion of edges to nodes within the phenotype (edges to other white nodes), for each white node in the graph (\frac{D_k}{nm});
calculate the average across all nodes in set K to reflect the relatedness and clustering of the realized genomes in the phenotype.

D_{K} =  {\frac{1}{k} \sum_{i=1}^{k}}\frac{D_k}{nm} 


Unrealized network robustness is an analogous measure to Phenotypic robustness () but focuses specifically on unrealized mutational robustness, not directly on realized neutral variation. This measure is therefore diagnostic of apparent variations in phenotypic frequency. 
Unrealized Network Robustness, , is measured by:
enumerate all white nodes (set K)
count the proportion of edges to potential mutants within the phenotype (edges to black nodes), for each white node in the graph (\frac{U_k}{nm});
calculate the average across all nodes in set K to reflect the level of mutational robustness in the population for a given phenotype.

U_{K} =  {\frac{1}{k} \sum_{i=1}^{k}}\frac{U_k}{nm} 


Network evolvability is an analogous measure to Genomic Diffusion Rate, and is used to relate the new network measures to existing measures of evolvability.
Network Evolvability, , is measured by:
enumerate all white nodes (set K)
for each white node, count the number of colored nodes (C)
for each white node, calculate the entropy of the edges to colored nodes, to identify the diversity of phenotypes available from that node: .
normalize the entropy against the total number of colored node edges to reflect the probability of a genotype connecting to a variety of other phenotypes ().
calculate the average across all nodes in set K to reflect the level of evolvability in the population for a given phenotype.

E_{K} =  {\frac{1}{k} \sum_{i=1}^{k}} E_k 


We will experimentally relate these measurements to the realized genomic diffusion rate, and to other measures of robustness to identify patterns in the mutational landscape that result in high evolvability while controlling for varying levels of phenotypic frequency.
