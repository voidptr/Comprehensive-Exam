\subsection{Proposed Methods}

In order to address the question of how the shape and density of neutral networks affects both robustness and evolvability, we will use Avida to generate a series of populations with known qualities with regard to robustness and evolvability. Then, we will analyze these populations with graph metrics to identify those features of the neutral networks that might promote robustness or evolvability.

Initially, we will use hill climbing to produce populations of solutions that evolve in a perfectly flat fitness landscape, and are therefore neither particularly robust nor evolvable. These populations will act as a control that allows us to produce baseline measures of neutral network density, network evolvability, and unrealized network robustness. We will also generate and measure random neutral networks as a further control. 

Next, we will generate a series of treatment populations whose environment is known to produce populations that vary in qualities such as modularity and robustness. For modularity, we will use sexual reproduction to produce populations that have high modularity, and a non-sexual two-task environment to produce low modularity. For robustness, we will use will use high-single-peak environments [ref survival of the flattest] to produce populations with high robustness, and changing environments to produce populations with low robustness. Each of these populations will otherwise share identical environments and mutation operators.

For each of the output populations, we will collect measures of phenotypic robustness, spatial modularity, and genomic diffusion rate (evolvability), as well as the neutral network metrics for evolvability. We will create a least-squares fit model to identify correlation between the graph metrics and the standard robustness, modularity, and evolvability metrics. Finally, we will identify the unique features of the neutral networks that correspond with varying levels of evolvability and robustness. 