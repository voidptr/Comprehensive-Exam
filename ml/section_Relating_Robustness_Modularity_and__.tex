\section{Relating Robustness, Modularity and Evolvability with  Mutational Neighborhood (Proposal)}

\subsection{Background}

Robustness and evolvability have long been considered to be at odds with each other. \verb|[ref]| Genotypic robustness, or the resistance of a genotype to phenotypic change via mutation, will reduce the incidence and impact of phenotype-altering mutations, whereas evolvability requires that mutations result in phenotypic change.

However, this conflict may be illusory. Robustness acts to increase the number of neutral genotypes that make up a phenotype within a population.\verb|[ref]| This spread of neutral variation may provide more and shorter paths to new phenotypes than if the phenotype only contained a few genotypes.\verb|[ref]| However, if that neutral variation is tightly clustered in a plateau, such that the genotypes that make up the phenotype have high individual mutational robustness, then overall evolvability for that phenotype is expected to be lower.\verb|[ref of center pushing]| In contrast, while holding neutral variation constant, we expect that phenotypes with low individual mutational robustness will exhibit higher evolvability. This is because those genotypes are spread more broadly along the mutational landscape, and are able to explore more of the phenotypic space with their mutants. \verb|[pull refs from Andreas Wagner]|

Therefore, rather than being either reinforcing or in opposition, the relationship between robustness and evolvability may be orthogonal, and mediated by the shape and density of the networks of neutral variants that make up the phenotype and the mutational landscape surrounding them. 

In this chapter, we will explore the relationship of different evolvable features with their realized phenotype graphs. We hypothesize that evolving phenotypes in conditions that increase modularity and lower pleiotropy, will result in more loosely connected phenotype graphs, while retaining high levels of phenotypic robustness.
