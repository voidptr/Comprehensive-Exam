\section{Modularity and Selective Mutational Fragility (Plasticity) Evolve in Changing Environments}
\subsection{Background}

Modularity is a commonly recurring motif in biological genomes. It contributes to both the robustness and evolvability of genomes in several ways. 1) It allows disparate functional components to evolve independently of each other. \verb|[ref, ref]| 2) Modularity allows for mutations that copy functional components (duplication), allowing for better repurposing of existing components.\verb| [ref, ref]| 3) Modularity enables the composition of those components into larger structures, to form robust gene regulatory networks \verb|[1,ref]|. 4) Modularity allows these networks to be deformed in response to mutation, allowing for the rapid evolution of altered function \cite{mody_modularity_2009} \verb|ref]|.

Despite its ubiquity, the evolutionary origins of modularity are not well understood. Many models of evolution produce highly non-modular structures \cite{kashtan_spontaneous_2005}. However, certain kinds of environments have been shown to encourage the evolution of modularity. sexual reproduction has been shown to increase measures of modularity in digital organisms \cite{misevic_sexual_2006}, in response to the risk of recombining incompatible functional segments. Changing environments have also been shown to encourage the evolution of modular neural networks \cite{kashtan_spontaneous_2005} as long as those environments select for modularized goals. 

Not all modular architecture must take place at the network level. One quasi-modular biological motif of interest is enzyme promiscuity, where internal organization allows it to promote secondary functions rapidly in response to selection, while maintaining its primary function. [\cite{aharoni_evolvability_2005}, \cite{khersonsky_enzyme_2006}] This mechanism depends on creating mutable regions of the module, where mutations can be tolerated without altering the primary phenotype. The presence of these regions thus simultaneously promotes both robustness and evolvability.

In order to examine the evolutionary pressures that encourage the evolution of modular and evolvable architecture, we subjected digital organisms to cyclical changing environments. Our research explores whether these shifting pressures result in different genetic architectures than what evolves under static conditions, and whether these architectures are more modular, and indeed more evolvable and robust than their static counterparts.