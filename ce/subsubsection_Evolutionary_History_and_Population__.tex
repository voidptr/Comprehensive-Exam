\subsubsection{Evolutionary History and Population Structure}

Organisms evolved in the hostile changing environment have a much longer phylogenetic history than those evolved in static or benign environments. At each environmental shifts, adaptive mutations rapidly sweep and fix in the populations.

Phylogenetic Depth - Figure

Caption: Phylogenetic depth of population evolved in the static (control) environment vs that of hostile-evolved. Sweeps are marked by white horizontal lines. Control-evolved has a maximum depth of 400 ancestors, while hostile-evolved has upward of 1100.

The populations evolved in the control and benign environment displayed much more genetic diversity as compared to the hostile environment, which underwent what was effectively a bottle-necking at each cycle shift. Despite the higher amounts of diversity in the benign and control treatments, some preserved regions are still identifiable.

Per-Site Population Entropy over Time - Figure

Caption: Frequent sweeps resulting from harsh environmental shifts result in very low population and per-site entropy in the hostile treatment. This is the result of bottle-necking at each cycle-shift.

