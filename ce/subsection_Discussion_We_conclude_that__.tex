\subsection{Discussion}

We conclude that changing environments can produce a kind of functional modularity in digital organisms by separating out unrelated portions of the genome in order to be able to quickly gain and lose a task via few mutations. This adjustment to their genetic architecture occurs as a result of the cycles of environmental shift increasing the number of mutations the genomes must accept, creating a reservoir of vestigial neutral sites, and plastic regions where mutations can more easily re-activate dormant function.