We conclude that changing environments can produce a kind of mutational plasticity in digital organisms by separating out unrelated portions of the genome in order to be able to quickly gain and lose a task via few mutations. This adjustment to their genetic architecture occurs as a result of the cycles of environmental shift increasing the number of mutations the genomes must accept, creating a reservoir of vestigial neutral sites, and plastic regions where mutations can more easily re-activate dormant function.

This functionality evolved de-novo, with genetic precursors that exhibited very low levels of modularity. This shows that quasi-modular structures can evolve without additional architectural scaffolds, and purely in response to a cyclically-changing environment. 