\subsubsection{Genetic Architecture}

These rapid shifts result in qualitatively different architectural styles. Due to the continuous acceptance of adaptive mutations, the genomes evolved under both experimental treatments are much more scattered, with the bulk of the sites responsible for performing the fluctuating task (EQU) separated from the backbone task (XOR), except for a core region of overlap, which represent portions of the tasks that are shared between XOR and EQU.
\begin{quote}
[ FIGURE WITH THE MAP OF THE DOMINANT ORGANISM]
\end{quote}
Interestingly, both the benign and harsh treatment-evolved populations also show development of a large reservoir of vestigial sites; that is, sites that were previously active in performing a task, but were disabled by a mutation elsewhere and are now neutral.  However, these vestigial sites do appear to be important for allowing the organisms to quickly readapt as the fluctuations in the environment restore the previously-rewarded functions.

Functional Site-Count - Figure

Caption: Number of functional vs vestigial sites by treatment. The hostile environment has a much larger number of vestigial sites compared to the benign or control, while having a comparable number of functional sites.

