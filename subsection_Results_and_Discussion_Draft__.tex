\subsection{Results and Discussion – Draft 70}

In all experimental treatments, we saw significantly reduced hybridization rates


Mating rates between populations is reduced to half of the within-population mating rates, indicating significantly reduced gene-flow. The mating rate is directly correlated to the difference in display traits between the populations, because it is the main factor in allowing females to discriminate between mates. 
Relatedly, we saw significant differences in the average display trait values between the isolated population.
FIGURE HERE
Interestingly, this change in display trait, and thus hybridization rates cannot be attributed to differential adaptation. 
The lack of statistically significant difference between the drift treatment and all other treatments indicates that drift is sufficient to generate difference in display trait, whereas the effect of differential adaptation is overwhelmed when compared to the effect of drift.
is there some other explanation? confounding effect?
We measured the fitness and viability of offspring in forced mating trials.




In all trials, we saw very little evidence of reduced viability or fitness of hybrids produced between populations, as compared to offspring from forced matings within populations.
This indicates that the architectures of the organisms did not vary sufficiently between populations to produce significantly less-fit offspring, despite varying sufficiently to produce variation in selected traits and choosiness.
Ultimately, this means that sexual selection can have a strong influence on a populations tendency to speciate, acting on short time-frames, and creating pre-zygotic isolation well before post-zygotic incompatibilities emerge to cement the status of new species.