Evolvability is a highly debated term, but most definitions can be summarized as the potential of populations and genomes to produce adaptive variation and complex structures in response to mutation and selection. Evolvability is thought to be created through the interplay of modularity and robustness. Indeed, modularity and robustness build upon each other, layer by layer, to form a framework that produces more powerful adaptive effects while reducing the likelihood and impact of deleterious mutations. In the first chapter, I provide a survey of current literature on evolvability, robustness, modularity, and how they are all believed to relate.

Despite their complex interdependence in ongoing evolutionary dynamics, the origins of modularity and robustness can be shown to evolve via simple mechanisms of direct selection. In the Chapter 2, we show how modular genetic architectures can evolve independently of other features, in response to a changing environment.

In the third chapter we propose to study how modularity might arise through horizontal gene transfer.  We hypothesize that organisms will choose to uptake gene fragments for food, even if there is a chance that fragments might be integrated into the genome.  Further, we expect that this uptake will result in higher modularity and evolvability over time.

In Chapter 4 we propose to study the effect of modularity and robustness on the character of the local mutational landscape.  We hypothesize that the single-step mutational landscape surrounding a given phenotype will have more neutral roads to different phenotypes if that phenotype is robust, and the genotypes in that phenotype are modular.

In the fifth chapter, we show that sexual selection facilitates speciation through reduced gene-flow.  We show that sexual selection is a strong early facilitator of reproductive isolation, and that post-zygotic isolation is not required to reduce gene flow between populations.
In the final chapter we include a project plan, including a timeline for completion of the remaining experiments.