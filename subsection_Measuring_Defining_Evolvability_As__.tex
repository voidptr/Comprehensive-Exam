\subsection{Measuring/Defining Evolvability}

As alluded to above, the features that confer robustness may also promote evolvability by allowing for greater neutral genetic diversity within a given phenotype. The larger the number of distinct genotypes with the same phenotype, the more exploration of the genotype space that can be done without decreasing organismal fitness. As a population diffuses through such a neutral region, more potential phenotypes become available in few mutational steps. [ref neutrality paper]
Historically, predicting this robust-yet-evolvable quality has been challenging. Previously-used measures for robustness that focus on counting the proportion of unique genotypes that compose a phenotype (Phenotypic Robustness [ref Andreas Wagner]) are limited in their ability to predict the evolvability of a population, especially where phenotypic frequency is unknown. 
In contrast, Genomic Diffusion Rate, or the probability that an offspring will express a neutral or positive fitness effect, may be used to characterize overall population evolvability [ref to evolvability in genomic diffusion rate]. However, while genomic diffusion rate is predictive of evolvability, it does not distinguish between types of robustness, and limited in its ability to mechanistically explain the evolvability characteristics of a phenotype in a given population. 
In order to address both of these issues, we will measure evolvability using a series of network graph metrics that describe both the level and shape of interconnectedness of realized neutral variants and their potential mutants. These aggregate metrics will be theoretically and experimentally related to existing robustness and evolvability measures to predict the evolvability of a population by accounting for both the probability of mutations changing phenotypes, as well as addressing the reasons behind those probabilities.


