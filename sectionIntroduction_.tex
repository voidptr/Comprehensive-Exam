\section{Introduction}
\section{Evolution of Modularity in Changing Environments}
\subsection{Background}
Genetic modularity contributes to both the robustness and evolvability of natural genomes by allowing disparate functional components to evolve independently of each other, and allowing for the composition of those components into robust gene regulatory networks \cite{Hartwell99}. However, the evolutionary origins of biological modularity are not well understood. While modularity of functional components is a commonly recurring motif in biological genomes, it appears to be a costly adaptation, which tends to evolve away under static environmental conditions \cite{Kashtan05}. Even so, research has show that sexual reproduction appears to increase the level of modularity in digital organisms (linear, self- replicating computer programs), as do changing environments that reward modular tasks in evolving neural networks \cite{Kashtan05}.

Under static environmental conditions, where selective pressures and environmental conditions remain fixed, digital organisms have a strong selective pressure to evolve tightly condensed and efficient genomes that exhibit a high degree of pleiotropy and epistasis. Related functional regions are condensed and overlapped, to achieve the execution of required functions with a minimal set of instructions.  However, under changing environments, the balance of selective pressures shifts toward favoring evolvability and adaptability rather than pure efficiency. Our research explores whether these shifting pressures result in different genetic architectures than what evolves under static conditions, and whether these architectures are more modular, and indeed more evolvable and robust than their static counterparts.
\subsection{Methods}
We used the Avida digital evolution platform to examine the effects of changing environments on the genomes of evolved digital organisms. Digital organisms are circular, self-replicating, and evolving computer programs. Populations of digital organisms are seeded from a single self-replicating ancestor into a 60x60 toroidal world, where the individuals compete with their peers for space to replicate. The organisms that can replicate the fastest come to dominate the environment. In order to replicate faster, the organisms may perform certain rewarded logical operations to increase the speed with which they execute their genomes. They may also shorten and streamline their genomes to reduce the number of instruction executions needed to replicate. The individual genomes are held at a fixed length of 121 instructions, but are mutated after each successful replication event at a rate of 0.00075 point mutation events per site. 

We subjected a total of 150 independent populations of digital organisms to two different types of two-phase cyclical changing environments: a benign changing environment, and a hostile changing environment, plus a static (non-changing) environment as a control (fig. \ref{evomod_env}).

The benign environmental treatment is characterized by fluctuating reward for performing the logical task EQU (the fluctuating task). This reward changes periodically in two phases, from zero reward, to a 25 merit bonus (which represents a multiplier on base organismal fitness of 25), and back to zero. The phases are of equal length, and repeat cyclically throughout the experiment. The organisms are also rewarded continually for performing a different logical task XOR (the backbone task), as a way of encouraging the organisms to preserve sections of their genome through the fluctuating periods, and thus have a basis for comparing the separation or intertwining of functional genetic components. The backbone task is continually rewarded by a 23 merit bonus. Each cycle lasts 1000 updates (roughly 30 generations), and the experiment lasts for 200 cycles per experimental run.

The hostile treatment is identical to the benign treatment except that the reward cycles between a -25 merit bonus (which acts as a strong punishment for performing the task during the off-cycle), and a positive 25 merit bonus.

The static environmental treatment rewards both tasks continuously at 25 (fluctuating task) and 23 (backbone task) merit bonuses.
\subsection{Results and Discussion}
Our experiments show that digital organisms that evolve in cyclic changing environments differ significantly from those evolved in static environments in a number of key ways.
\subsubsection{Evolutionary history and genetic architecture}
Organisms evolved in the hostile changing environment have a much longer phylogenetic history than those evolved in static or benign environments. At each environmental shifts, adaptive mutations rapidly sweep and fix in the populations. (figs. \ref{evomod_phylo}, \ref{evomod_entropy})
