\subsection{Methods}
We propose to implement HGT within Avida (see Figure \ref{hgt_ process}). As organisms die, their genomes will be accumulate in reservoirs within world-cells. These reservoirs will age-out, with older genomes disappearing from the reservoir as new ones are accumulated. Fragments for uptake will be randomly selected from the reservoir. Two new mutagenic instructions will be created to trigger uptake of fragments. One of the instructions will confer an energy bonus upon uptake, while the other will not. Both instructions will incur a random chance of incorporating the fragment as an insertion into the genome. If the insertion occurs, no energy bonus will be given. The energy bonus level, uptake mutagenic chance, and fragment size will be configurable per experiment.
There are two major hypotheses about HGT usage. First, that HGT usage is a random side-effect of grazing for energy sources[ref]. The second is that HGT is an evolved mechanism for increasing evolvability [ref]. We will test these two hypotheses by giving organisms an opportunity use HGT in various contexts that may or may not reward evolvability.


Hypothesis 1: Organisms will uptake gene fragments from the environment at a higher rate if they are useful as an energy source, even if such an uptake increases the chance of mutation. Without an energy bonus, gene fragment uptake will remain at low frequency.


In static environments, evolvability pressures are expected to be low, therefore the primary benefit of gene fragment uptake is expected to be from the energy received. Without this energy bonus, the disruptive effect of integrating random gene fragments into your genome would result in reduced fitness, and thus lower frequency of uptake in the population.


Hypothesis 2: Organisms will uptake gene fragments from the environment at a higher rate in environments where evolvability is beneficial, such as in changing environments. In static environments, uptake frequency will remain lower.


Hypothesis 2b: In changing environments, organisms will still uptake gene fragments at higher frequencies, even with reduced, or non-existent bonus reward from uptake.


In environments where evolvability pressures are high, such as changing environments, increased mutation rates are selected for, whereas in static environments, high mutation rates are selected against[ref]. HGT-mediated mutations are therefore expected to be selected for in changing environments, vs static environments. Further, in changing environments, the energy bonus from uptake may not be neccessary to ensure fragment uptake.


\begin{mdframed}
-- BELOW IS PULLED FROM BEACON PROPOSAL - NOT INCLUDED IN FINAL TEXT --


We propose to implement HGT within Avida (see Figure 1). As organisms die, their genomes will accumulate in reservoirs. As older genomes are displaced by newer ones, fragments for uptake will be available from the reservoir. We will create three new mutagenic instructions that trigger uptake of these fragments. The first of the instructions, Uptake-HGT-Bonus, will allow an organism to ingest a fragment to metabolize for a fitness bonus, with a chance of recombining it into its genome instead, paralleling an existing theory from the biological literature \cite{Redfield2001}. The second instruction, Uptake-HGT-noBonus, will allow ingesting the fragment, but not metabolizing, while the third instruction, Uptake-noHGT-Bonus, allows metabolizing the fragment, but no chance for HGT. The magnitude of the fitness bonus, the chance of mutagenic uptake, and the fragment size will be configurable per experiment. 

Our first set of hypotheses examine the some of the environmental conditions that might promote HGT 

Hypothesis 1a: Organisms will uptake gene fragments from the environment if they provide a fitness benefit, even if such an uptake has a chance of disrupting their genome. 

Hypothesis 1b: Without a fitness benefit, gene fragment uptake will remain at low frequency. 

Hypothesis 1c: All else equal, evolved organisms will not risk HGT if an equivalent alternative is available. 

Many microbes exhibit natural competence, in which DNA is ingested from the environment as a growth medium. DNA is a source of carbon, nitrogen, and phosphorous, as well as many high-energy bonds that can be metabolized. This natural competence comes with a potential cost, in that the DNA may recombine with that of the ingesting organism, thereby disrupting important genes or bringing in harmful ones. It is typically assumed that the benefit of foreign DNA as an energetic resource outweighs these costs for the organisms that have evolved natural competence \cite{Redfield2001}. 

To test this principle, we will add Uptake-HGT-Bonus to the list of instructions that can result from mutations in Avida, as well as nop-X, a control instruction that does not affect the executing organism or the environment when executed. We will test whether Uptake-HGT-Bonus is used at a higher rate in evolved populations than nop-X, which would indicate that there is an advantage to using it, rather than it existing at low levels due to mutation alone. Conversely, if an instruction is used significantly less frequently than nop-X, it would indicate an active selective disadvantage associated with that instruction. 

We will then replace the Uptake-HGT-Bonus instruction with Uptake-HGT-noBonus, which provides the opportunity for HGT without providing a metabolic benefit to the executing organism. We expect to see a lower rate of use of Uptake-HGT-noBonus than we saw of Uptake-HGT-Bonus, potentially no greater than nop-X. If we find that organisms do use Uptake-HGT-Bonus at greater than nop-X frequency, we will then replace Uptake-HGT-Bonus with Uptake-noHGT-Bonus, which provides CPU cycles but has no chance of causing HGT. We will then compare the frequency in evolved genomes of Uptake-noHGT-Bonus to the frequency of Uptake-HGT-Bonus. If Uptake-HGT-Bonus is used more often than Uptake-noHGT-Bonus, that would provide strong evidence that there is an advantage to the HGT component of the instruction independent that of the CPU cycle bonus. 

Hypothesis 2: Genomic fragments that contain a complete task in a shorter physical length will be more likely to be incorporated by HGT than fragments that complete a task in a longer physical length. Within evolved Avida populations, the genomic length of a segment used to complete a given task can vary substantially \cite{ComplexFeat}. During recombination, sites that are further apart are more likely to become disassociated from each other than are sites closer together. Because our method of HGT involves recombination of a fragment of a donor genome and the recipient?s genome, we predict that donor genomes that encode a given task in a smaller genomic length will be more likely to be incorporated by HGT. For one, smaller genomic lengths needed for a task mean that a higher proportion of fragments of a given length will contain the complete task. For another, even if a fragment contains an entire task, recombination between a fragment and a host genome does not always occur at the edges of a fragment, and thus a more compact task will be less likely to be disrupted by recombination. 

To test this hypothesis, we will perform additional experiments where we seed the fragment reservoirs, rather than letting them fill from the current population. The genome fragments in one set of runs will contain compact tasks, while fragments in another set of runs will come from organisms with highly dispersed tasks. We will compare the evolved organisms in these runs to each other and to runs using the previous methodologies where reservoirs were filled with fragments of dead organisms. 

Hypothesis 3: Activation of HGT mechanisms will result in evolved genomes with a higher degree of physical modularity than populations without HGT. 

Bacterial genomes are strikingly modular. In addition to being compact, with low levels of non-coding DNA, bacterial genomes are typically arranged into operons, where several genes required for a particular function are physically grouped together \cite{Yin2010}. This differs from most eukaryotic genomes, where many functions require a series of gene products that are encoded by sites scattered throughout the genome. It has been hypothesized that HGT may cause this modularity, because genomes which happen to have the needed genes for a function grouped physically together will be more likely to spread this grouping to other organisms via HGT than organisms in which the needed genes are more dispersed through the genome \cite{Deem2013,Munoz2008}. We will test this theory by measuring the physical modularity of genomes from environments with HGT and environments without HGT with the same overall mutation rate.
\end{mdframed}
