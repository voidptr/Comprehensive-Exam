\subsubsection{Origins of HGT in nature}

In naturally evolving organisms, “natural competence” is an HGT mechanism by which organisms uptake the DNA of dead organism in the environment. These organisms benefit in several ways. DNA is composed of [phosphorus, etc, fill this in], materials that are useful for DNA synthesis and repair. The organisms may also benefit from uptaking gene-fragments that confer new adaptive functionality into the genome [ref]. However, it is unclear whether the origins of these functions were developed in order to obtain nutrients, or if the acquisition of new functionality was directly selected for. [ref Rosie redfield]. While grazing for gene-fragments as nutrients certainly conveys an advantage, the possibility of integrating these gene fragments may be more disruptive than beneficial to organisms. Even so, HGT is a wide-spread phenomenon in nature.
In this chapter, I propose to study environmental pressures that promote the use of horizontal gene transfer, and attempt to disambiguate between the grazing and new-function hypotheses. I also propose to study the effects of HGT on genetic architecture, including modularity and robustness.