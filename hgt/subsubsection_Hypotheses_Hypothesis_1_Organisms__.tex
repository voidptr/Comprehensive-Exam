\subsubsection{Hypotheses}

The first series of hypotheses address the 

Hypothesis 1a: Organisms will uptake gene fragments from the environment at a higher rate if the fragments are useful as an energy source, even if such an uptake increases the chance of mutation. 

Without an energy bonus, gene fragment uptake will remain at low frequency. In static environments, where evolvability pressures are expected to be low, the primary benefit of gene fragment uptake is expected to be from the energy received. Without this energy bonus, the disruptive effect of integrating arbitrary gene fragments into an organism’s genome would result in reduced fitness, and thus select for a lower frequency of uptake in the population.

Hypothesis 1b: Organisms will uptake gene fragments from the environment at a higher rate in environments where evolvability is beneficial, such as in changing environments. In static environments, uptake frequency will remain lower. We expect that the benefit from uptaking fragments that confer the desired functionality will outweigh the risk of genomic disruption.

Hypothesis 1c: In changing environments, organisms will still uptake gene fragments at higher frequencies, even with reduced, or non-existent energy gain from uptake. We expect that even without an energy bonus, that the benefit gained from integrating desirable functionality will keep uptake frequency higher than in static environments.