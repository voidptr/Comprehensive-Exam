\section{Speciation via Sexual Selection, and the effect on Adaptation and Genetic Architecture – Draft 50}

\verb|This feels like a pretty tenuous connection - I’m going to need to figure out how to reformulate this to make sense.|

\subsection{6.1 Background}

Sexual selection is a sexually-reproductive selection mechanism in which one sex (generally the higher-carer) choose their mates based on preferred values of one more traits \verb|[ref]|. Those of the other sex then compete for mates on the basis of that trait. Those with the most prominent trait expression tend to have better mating success, passing on the trait to their offspring, and driving greater and greater expression of that trait. 

Sexual selection is ubiquitous in nature, however selection can drive some trait values to such extremes that expressions of those traits becomes detrimental to condition and survival. Because of this apparent contradiction, there are many competing hypotheses accounting for the origins and maintenance of sexual selection, including the good genes hypothesis \verb|[ref]|, the sexy sons hypothesis \verb|[ref]|, or, for random-seeming traits, even Fisherian runaway trait selection \verb|[ref]|. 

Sexual selection is thought to contribute to speciation by allowing mates to discriminate against hetero-specifics during courtship. This discrimination would allow the choosers to avoid producing unfit offspring because of genetic incompatibilities or maladaptive trait combinations, or even by failing to reproduce altogether. In this way, sexual selection is thought to contribute to creating and reinforcing species boundaries.

Adaptive radiations and speciation are thought to be one mechanism by which evolvability innovations spread into new niches \cite{kirschner_evolvability_1998}\cite{dawkins_13_2003}. Those organisms with evolvable traits may more easily be able to colonize new niches, and drift in the traits that are sexually selected for may allow the evolvable traits to remain and fix within the founding population, without dilution by mating with less-evolvable relatives. 

In this chapter, I will discuss research on the effects of sexual selection on gene-flow between temporarily isolated populations, and compare the effects of mate-choice-mediated pre-zygotic isolation against post-zygotic effects.