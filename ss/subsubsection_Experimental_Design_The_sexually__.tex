\subsubsection{Experimental Design}

The sexually-reproducing Avidians are placed in an environment where they are given a fitness bonus for performing a pre-determined set of logical tasks.  The composition of these initial rewarded tasks varied based on the treatment type.

The organisms are allowed to evolve in this environment for 100,000 updates, or between 1,500 and 3,000 generations. Then, we divide the population into two isolated halves, and allow them to evolve independently for an additional 100,000 updates.

The environments in each half of the isolated populations are different from each other. The first treatment varies the halves by restricting the tasks available for reward, with each side adapting to a different set of restricted options. The second, third, and fifth treatments varies the halves by supplying a different expanded set of tasks for reward, with each side adapting to a different set of more complex tasks. The fourth treatment does not vary the tasks available in the halves, thus any change between the halves could be attributed to drift. The sixth condition is a control, where the population is never divided, but allowed to evolve for 100k updates together.

After this period of isolation, the wall between the organisms is removed, and they are allowed a single round of matings. We measured the rate of hybridization based on this initial interaction, and compared it to the rate of matings that occurred within groups. We also performed forced matings between and within the populations and measured the resulting offspring’s fitness and viability.