\subsubsection{Evolvability as a Distinct Concept}

Dawkins, in his foundational paper on evolvability and evolutionary constraint [ref], reframed the problem of evolvability in the context of computational evolution and development.  Dawkins described a generative genetic system based on a few alleles, and rules that governed development based on the traits encoded in the alleles.  Each allele would govern the execution of a generative rule, and the rules would interact with each other as they produced the phenotype. As he added new kinds of rules (constraints) into the generative process, the system would produce more and more complexity.

Dawkins used this example to draw parallels to biological generative developmental systems and how evolutionary constraints in development allow for more complex and robust phenotypes. Dawkins identified a few key themes that underlay the more powerful features of developmental systems. These systems would be organized in such a way as to facilitate cumulative effects. That is, innovations in constraints can build upon each other and are “cumulative in evolutionarily interesting ways”[ref]

Dawkins theorized that these kinds of generative developmental systems, or “embryologies” were the basis for evolvability, and that they must have evolved as a result of their intrinsic power to produce adaptive variation. Dawkins suggested that the genetic systems that persisted were those that facilitated adaptive radiations into new or otherwise empty ecological niches.

Alberch followed up Dawkins’ ideas with a much more thorough accounting of how, exactly, these kinds of evolvable traits translate into an analyzable phenotype space. Alberch dismantled the concept of a simplistic, hierarchical genotype to phenotype mapping function and emphasized that developmental and cell metabolic systems are strongly dynamical, nonlinear systems, for which genes are just one part of the regulatory cycle. Because of the dynamic nature of cell processes, it was clear that the gene-centric, population genetics view was inadequate to describe the complexity of the processes involved, and how they translated complex parameters into phenotypes. To that end, a new framework for analysis was required.

Alberch introduced the concept of “parameter spaces” to describe the variation in genotypic parameters that result in distinct phenotypes, while addressing the lack of one-to-one correlation between alleles (parameters) and phenotype. Parameter spaces are multidimensional spaces, divided by parameter thresholds (bifurcation boundaries) that form boundaries between phenotypes.  The domains bounded by these thresholds include all of the parameter combinations that produce a given phenotype. Larger domains can be described as more stable than smaller domains, because there are larger ranges of neutral variation available before organisms tip into a different phenotype. Populations with distinct phenotypes and varying parameters can thus be visualized as blobs occupying areas in parameter space.

Alberch contended that the “evolvability potential” of a dynamical system is encapsulated by the properties of the parameter space. Specifically, the topology of the bifurcation boundaries govern the ease with which the systems can produce both neutral and adaptive variation. Alberch asserted that the generative systems must have undergone selection that favors those systems that provide a good balance between exploration and stability, but provided no mechanism for that selection.

Dawkins and Alberch laid out a compelling case for the role generative developmental systems in facilitating evolvability, but their theoretical frameworks were far from complete.

