\subsection{Conclusions}
We conclude that changing environments can indeed produce a kind of functional modularity in digital organisms by separating out unrelated portions of the genome in order to be able to more quickly gain and lose a task. This occurs as a result of the cycles of environmental shift increasing the number of mutations the genomes must accept creating a reservoir of usable neutral mutations. However, it is not clear that this kind of genetic organization is sufficient to produce better long-term evolvability or evolutionary potential.
