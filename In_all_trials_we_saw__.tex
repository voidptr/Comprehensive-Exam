In all trials, we saw very little evidence of reduced viability or fitness of hybrids produced between populations, as compared to offspring from forced matings within populations.

This indicates that the architectures of the organisms did not vary sufficiently between populations to produce significantly less-fit offspring, despite varying sufficiently to produce variation in selected traits and choosiness.

We conclude that sexual selection can have a strong influence on a populations tendency to speciate, acting on short time-frames, and creating pre-zygotic isolation well before post-zygotic incompatibilities emerge to cement the status of new species.

In terms of the effect on the long term evolution of evolvability, features that contribute to rapid speciation, particularly adaptive radiation, may have contributed to the spread of evolvable features.